% Document setup
\documentclass[article, a4paper, 11pt, oneside]{memoir}
\usepackage[utf8]{inputenc}
\usepackage[T1]{fontenc}
\usepackage[UKenglish]{babel}

% Document info
\newcommand\doctitle{Dynkin systems and independence}
\newcommand\docauthor{Danny Nygård Hansen}

% Formatting and layout
\usepackage[autostyle]{csquotes}
\usepackage[final]{microtype}
\usepackage{xcolor}
\frenchspacing
\usepackage{latex-sty/articlepagestyle}
\usepackage{latex-sty/articlesectionstyle}

% Fonts
\usepackage[largesmallcaps,partialup]{kpfonts}
\DeclareSymbolFontAlphabet{\mathrm}{operators} % https://tex.stackexchange.com/questions/40874/kpfonts-siunitx-and-math-alphabets
\linespread{1.06}
\let\mathfrak\undefined
\usepackage{eufrak}
\usepackage{inconsolata}
\usepackage{amssymb}

% Hyperlinks
\usepackage{hyperref}
\definecolor{linkcolor}{HTML}{4f4fa3}
\hypersetup{%
	pdftitle=\doctitle,
	pdfauthor=\docauthor,
	colorlinks,
	linkcolor=linkcolor,
	citecolor=linkcolor,
	urlcolor=linkcolor,
	bookmarksnumbered=true
}

% Equation numbering
\numberwithin{equation}{chapter}

% Footnotes
\footmarkstyle{\textsuperscript{#1}\hspace{0.25em}}

% Mathematics
\usepackage{latex-sty/basicmathcommands}
\usepackage{latex-sty/framedtheorems}
\usepackage{latex-sty/probabilitycommands}
\usepackage{tikz-cd}
\usetikzlibrary{babel}

% Lists
\usepackage{enumitem}
\setenumerate[0]{label=\normalfont(\arabic*)}

% Bibliography
\usepackage[backend=biber, style=authoryear, maxcitenames=2, useprefix]{biblatex}
\addbibresource{references.bib}

% Title
\title{\doctitle}
\author{\docauthor}


% Section style -- add to section style .sty?
\setsubsecheadstyle{\normalfont\itshape}


% Preimage -- to be added to mathcommands .sty
\newcommand{\preim}{^{-1}}


\newcommand{\calN}{\mathcal{N}}
\DeclarePairedDelimiter{\nhoodfilteraux}{(}{)}
% \newcommand{\nhoodfilter}[1]{\calN\nhoodfilteraux{#1}}
\newcommand{\nhoodfilter}[1]{\calN_{#1}}


\newcommand{\calU}{\mathcal{U}}
\newcommand{\calV}{\mathcal{V}}
\newcommand{\calW}{\mathcal{W}}
\newcommand{\calT}{\mathcal{T}}
\newcommand{\calB}{\mathcal{B}}
\newcommand{\calE}{\mathcal{E}}
\newcommand{\calF}{\mathcal{F}}
\newcommand{\calA}{\mathcal{A}}
\newcommand{\calD}{\mathcal{D}}
\newcommand{\calS}{\mathcal{S}}

\newcommand{\borel}[1]{\calB(#1)}
\DeclareMathOperator{\supp}{supp}

\let\oldP\P
\renewcommand{\P}{\mathbb{P}}

\begin{document}

\maketitle


\chapter{Introduction}

These notes are basically a proof of Theorem~6.3 in Bauer.


\chapter{Dynkin systems}

\begin{definition}[Dynkin systems]
    A collection $\calD$ of subsets of a set $X$ is called a \emph{Dynkin system} in $X$ if
    %
    \begin{enumdef}
        \item $X \in \calD$,
        \item $B \setminus A \in \calD$ for $A,B \in \calD$ with $A \subseteq B$, and
        \item $\bigunion_{n\in\naturals} A_n \in \calD$ for every increasing sequence $(A_n)_{n\in\naturals}$ of sets in $\calD$.
    \end{enumdef}
\end{definition}
%
A Dynkin system is also variously called a Dynkin class, $\delta$-system, $d$-system, or $\lambda$-system. Clearly every $\sigma$-algebra is a Dynkin system.

The motivation for considering Dynkin systems is twofold. First of all they are significantly simpler than $\sigma$-algebras, and working with Dynkin systems instead of $\sigma$-algebras can often we done with no loss of generality, as the following fundamental result shows:

\begin{theorem}[Dynkin's Lemma]
    Let $\calS$ be a collection of subsets of a set $X$ that is closed under finite intersections. Then
    %
    \begin{equation*}
        \delta(\calS) = \sigma(\calS).
    \end{equation*}
\end{theorem}
%
Also known as \textquote{Dynkin's $\pi$-$\lambda$ theorem} since a non-empty collection of sets that is closed under finite intersections is also called a $\pi$-system.

\begin{proof}
    Bauer Theorem~2.3 (different definition of Dynkin systems), Cohn Theorem~1.6.2.
\end{proof}

Another source of motivation comes from the following result about finite measures which is important when proving uniqueness of properties of finite or $\sigma$-finite measures:

\begin{lemma}
    \label{thm:finite-measures-agree-on-Dynkin}
    Let $\mu$ and $\nu$ be finite measures on a measurable space $(X,\calE)$ such that $\mu(X) = \nu(X)$. The family $\calD \subseteq \calE$ of sets on which $\mu$ and $\nu$ agree is a Dynkin system.
\end{lemma}

\begin{proof}
    By assumption $X \in \calD$. Let $A_1, A_2 \in \calD$ with $A_1 \subseteq A_2$. Then
    %
    \begin{equation*}
        \mu(A_2 \setminus A_1)
            = \mu(A_2) - \mu(A_1)
            = \nu(A_2) - \nu(A_1)
            = \nu(A_2 \setminus A_1),
    \end{equation*}
    %
    since $\mu$ and $\nu$ are finite. Finally assume that $(A_n)_{n\in\naturals}$ is an increasing sequence of elements in $\calD$. Then by continuity we have
    %
    \begin{equation*}
        \mu \Bigl( \bigunion_{n\in\naturals} A_n \Bigr)
            = \lim_{n\to\infty} \mu(A_n)
            = \lim_{n\to\infty} \nu(A_n)
            = \nu \Bigl( \bigunion_{n\in\naturals} A_n \Bigr).
    \end{equation*}
    %
    Thus $\calD$ is a Dynkin system as claimed.
\end{proof}


\chapter{Independence}

Below we let $(\Omega, \calF, \P)$ be a fixed probability space.

\begin{definition}[Independence I]
    Let $I$ be a non-empty index set. A family $(A_i)_{i \in I}$ of events from $\calF$ is called \emph{independent} relative to $\P$ if
    %
    \begin{equation*}
        \P \Bigl( \bigintersect_{\nu = 1}^n A_{i_\nu} \Bigr)
            = \bigprod_{\nu = 1}^n \P(A_{i_\nu}),
    \end{equation*}
    %
    for any finite subset $\{ i_1, \ldots, i_n \}$ of distinct elements of $I$.
\end{definition}


\begin{definition}[Independence II]
    Let $(\calA_i)_{i \in I}$ be a family of sets $\calA_i \subseteq \calF$ of events. This family is called \emph{independent} if
    %
    \begin{equation*}
        \P \Bigl( \bigintersect_{\nu = 1}^n A_{i_\nu} \Bigr)
            = \bigprod_{\nu = 1}^n \P(A_{i_\nu}),
    \end{equation*}
    %
    for any choice of events $A_{i_\nu} \in \calA_{i_\nu}$ and any finite subset $\{ i_1, \ldots, i_n \}$ of distinct elements of $I$.
\end{definition}
%
Notice that this definition reduces to the first one if all $\calA_i$ are singletons.

\begin{remark}
    \label{rem:finite-subfamilies-independent}
    Finite subfamilies independent is sufficient.
\end{remark}

\begin{proposition}
    Let $(\calA_i)_{i \in I}$ be an independent family of sets of events from $\calF$. Then the family $(\delta(\calA_i))_{i \in I}$ is also independent. In particular, if the $\calA_i$ are closed under intersection, the family $(\sigma(\calA_i))_{i \in I}$ is independent.
\end{proposition}

\begin{proof}
    By \cref{rem:finite-subfamilies-independent} we may assume that $I$ is finite.

    Fix an index $i_0 \in I$, and choose sets $A_{i_\nu} \in \calA_{i_\nu}$ for distinct indices $i_1, \ldots, i_n \in I \setminus \{i_0\}$. Then define measures $\P_1$ and $\P_2$ on $\calF$ by
    %
    \begin{equation*}
        \P_1(A)
            = \P \Bigl( A \intersect \bigintersect_{\nu = 1}^n A_{i_\nu} \Bigr)
        \quad \text{and} \quad
        \P_2(A)
            = \P(A) \bigprod_{\nu = 1}^n \P(A_{i_\nu}),
    \end{equation*}
    %
    for $A \in \calF$. Notice that $\P_1$ and $\P_2$ agree on $\calA_{i_0}$ by independence, so since $\P_1(\Omega) = \P_2(\Omega)$, \cref{thm:finite-measures-agree-on-Dynkin} implies that they also agree on $\delta(\calA_{i_0})$. It follows that
    %
    \begin{equation*}
        \P \Bigl( A \intersect \bigintersect_{\nu = 1}^n A_{i_\nu} \Bigr)
            = \P(A) \bigprod_{\nu = 1}^n \P(A_{i_\nu})
    \end{equation*}
    %
    for all choices of sets $A_{i_\nu} \in \calA_{i_\nu}$. But this precisely expresses the independence of the family $(\calA_i)_{i \in I}$ with $\calA_{i_0}$ replaced by $\delta(\calA_{i_0})$.

    Performing a finite number of such replacements, once for each index in $I$, proves the first claim. The second claim follows by Dynkin's lemma.
\end{proof}

% References: Cohn, Bauer, Bauer

\end{document}